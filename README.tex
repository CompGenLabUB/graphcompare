\documentclass[10pt]{article}
\usepackage[margin=1in]{geometry}
\usepackage{amsmath}
\usepackage[usenames,dvipsnames]{xcolor}
\usepackage{listings}
\usepackage{array}
\usepackage{graphicx}
\usepackage{wrapfig}
\usepackage{caption}
\usepackage{hyperref}

\title{Readme - dotcompare}

\author{}
\date{}



\begin{document}
\maketitle


\section{NAME}\label{name}

dotcompare - A program to compare DOT files

\section{VERSION}\label{version}

v0.1.4

\section{SYNOPSIS}\label{synopsis}

\begin{verbatim}
dotcompare  --files file1.dot,file2.dot \\  
            --colors HARD               \\   
            --dot output.dot            \\   
            --table table.tbl           \\ 
            --venn venn.svg             \\ 
            --web graph.html               
\end{verbatim}

\section{DESCRIPTION}\label{description}

This script compares two or more DOT (graphviz) files and prints the
resulting merged DOT file with different colors for each group.

By default, dotcompare will print the resulting graph to STDOUT, but you
can change it with the option -d (see options below).

Dotcompare has some optional outputs, each one specified by one option.

\begin{itemize}
\item
  \begin{itemize}
  \itemsep1pt\parskip0pt\parsep0pt
  \item
    Venn diagram.
  \end{itemize}

  If given the option -v, dotcompare will create an svg file containing
  a venn diagram. In this image, you will be able to see a comparison of
  the counts of nodes and relationships in each input DOT file, and
  those nodes/relationships common to more than one file. The colors
  will be chosen using one of the profiles in data/colors.txt. By
  default, the color palette is set to be ``SOFT''. To change it, use
  the option -c (see options below).
\item
  \begin{itemize}
  \itemsep1pt\parskip0pt\parsep0pt
  \item
    Table.
  \end{itemize}

  Complementary to the venn diagram, one can choose to create a table
  containing all the counts (so it can be used to create other plots or
  tables). The table is already formated to be used by R. Load it to a
  dataframe using:

\begin{verbatim}
    df <-read.table(file="yourtable.tbl", header=FALSE)
\end{verbatim}
\item
  \begin{itemize}
  \itemsep1pt\parskip0pt\parsep0pt
  \item
    Webpage with the graph.
  \end{itemize}

  With the option -w, one can create a webpage with a representation of
  the merged graph (with different colors for nodes and relationships
  depending on their presence in each DOT file). To make this
  representation, dotcompare uses the Open Source library cytoscape.js.
  All the cytoscape.js code is embedded in the html file to allow
  maximum portability: the webpage and the graph work without any
  external file/script dependencies. This allows for an easy upload of
  the graph to any website.
\end{itemize}

\section{OPTIONS}\label{options}

\begin{itemize}
\item
  \textbf{-h}, \textbf{--help}

  Shows this help.
\item
  \textbf{-f}, \textbf{--files}
  \textless{}file1,file2,\ldots{}\textgreater{}

  REQUIRED. Input DOT files, separated by commas.
\item
  \textbf{-d}, \textbf{--dot} \textless{}filename.dot\textgreater{}

  Creates a merged dot file. Default to STDOUT.
\item
  \textbf{-c}, \textbf{--colors} \textless{}profile\textgreater{}

  Color profile to use: SOFT (default), HARD, LARGE or CBLIND.
\item
  \textbf{-v}, \textbf{--venn} \textless{}filename.svg\textgreater{}

  Creates a venn diagram with the results.
\item
  \textbf{-w}, \textbf{--web} \textless{}filename.html\textgreater{}

  Writes html file with the graph using cytoscape.js
\end{itemize}

\section{INSTALLATION}\label{installation}

To install dotcompare you have two options: either you move the files
manually to wherever you want or you use the script \texttt{install.sh}.

If you use \texttt{install.sh}, it will ask you in which directory you
want to store the program and all the files it needs. You will need
\textbf{ROOT} privileges to use install.sh, as it creates a symlink to
dotcompare.pl in \texttt{/usr/local/bin} and a man page in
\texttt{/usr/share/man/man1/}.

\section{AUTHOR}\label{author}

Sergio Castillo Lara - s.cast.lara@gmail.com

\section{BUGS AND PROBLEMS}\label{bugs-and-problems}

\subsection{Current Limitations}\label{current-limitations}

\begin{itemize}
\item
  \emph{Undirected\_graphs}

  Only works with directed graphs. If undirected, dotcompare considers
  it to be directed.
\item
  \emph{Clusters}

  Still no clusters support eg: \{A B C\} -\textgreater{} D
\item
  \emph{Multiline IDs}

  No support for multiline IDs (yet).
\end{itemize}

\subsection{Reporting Bugs}\label{reporting-bugs}

Report Bugs at \emph{https://github.com/scastlara/dotcompare/issues}
(still private)

\section{COPYRIGHT}\label{copyright}

\begin{verbatim}
(C) 2015 - Sergio CASTILLO LARA

This program is free software; you can redistribute it and/or modify
it under the terms of the GNU General Public License as published by
the Free Software Foundation; either version 2 of the License, or
(at your option) any later version.

This program is distributed in the hope that it will be useful,
but WITHOUT ANY WARRANTY; without even the implied warranty of
MERCHANTABILITY or FITNESS FOR A PARTICULAR PURPOSE. See the
GNU General Public License for more details.

You should have received a copy of the GNU General Public License
along with this program; if not, write to the Free Software
Foundation, Inc., 675 Mass Ave, Cambridge, MA 02139, USA.
\end{verbatim}

\end{document}